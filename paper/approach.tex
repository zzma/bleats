\section{Approach}

\label{sec:approach}

BLEATS was designed to address the characteristics of speed, deployment ease,
authentication accuracy, reliability, and scale in CAT systems while providing
the main functions that are necessary in an attendance tracking solution of
recording identity, matching location, and noting the time. The basic BLEATS
functionality involves a student using BLE beaconing on their mobile device to
communication to a central instructor device which marks the student as
present. This basic BLEATS implementation addresses the challenges of speed and
deployment ease by nature of choosing BLE as the platform of communication as
BLE avoids the need for a lengthy association and is becoming evermore
ubiquitous. The basic BLEATS functionality also fulfills the first and second
functions of attendance tracking solutions, recording identity and establishing
location. By choosing BLE as the platform of communication, BLEATS has a
limited, but non-restrictive range that can be used to determine with
confidence that a user is within the classroom and that a beacon belongs to a
user by observing the beacon’s UUID.

The basic BLEATS functionality described above achieves the desired
characteristics of speed and deployment ease,  as well as providing the
identity and locality functions. However, three characteristics and one more
function are required by BLEATS. The rest of this section will describe how
these characteristics and functions are addressed through the BLEATS
authentication scheme and the BLEATS simulation.

\subsection{Authentication Scheme}

The BLEATS system requires an authentication scheme to ensure that a student’s
identity is as claimed to be. The authentication scheme should address the
characteristics of authentication accuracy and reliability and the attendance
tracking functions of recording identity and noting the time. It was determined
that a dual authentication scheme should be utilized for the BLEATS
authentication scheme. The reasoning behind this decision was that the basic
BLEATS implementation provides a single source of authentication through BLE
beaconing. The BLE beaconing protocol requires a UUID be transmitted with the
beacon and as such, a unique UUID assigned to each student acts as a first
source of authentication. However, this source of authentication and function
of recording identity does not provide optimal authentication accuracy as a
student could easily transmit a UUID that is not their own. Due to the
possibility of a spoofed beacon message, dual authentication was determined to
be the optimal method to ensure identity. 

The BLE beaconing from the basic BLEATS functionality is the first form of
authentication for the BLEATS authentication scheme, but several varieties of
authentication were considered for the second form of authentication. The first
of the alternative forms of authentication was GPS. GPS was considered due to
the function of GPS to further support identity by locality by ensuring that a
student is in the classroom. However, this option was discarded due to GPS’s
property of relatively high power consumption as well as the locality identity
already being addressed by the first BLEATS BLE beaconing. Additionally, GPS
coordinates could easily be spoofed by transmitting GPS coordinates that
correspond to the classroom to the server. The second of the alternative forms
of authentication was Wi-Fi authentication by requiring that a student connect
to a nearby access point and then ensuring that the student was in fact
connected to that access point. However, this was discarded due to a similar
reason as GPS in that locality identity was already considered by BLE
beaconing. Also similarly to GPS, Wi-Fi authentication could be spoofed by
other students in the classroom by connecting an alternate device to the Wi-Fi
access point that appears to be other students. A third alternative form of
authentication that was considered was email dual authentication. This was
discarded due to the reliance on Wi-Fi or mobile data reliability that was a
detriment described for web based applications previously.  After weighing
these considerations, it was determined that the optimal second form of
authentication would be sending a one time code through SMS messaging to the
student’s mobile device and requiring that they send it back to the server
through BLE beacon. This method was determined to be optimal due to providing
an alternate communication channel of SMS that would provide a two distinct
channels of authentication. Also, a student’s mobile phone could be used for
both the first and second forms of communication seamlessly. By requiring that
the student send the code back to the server through BLE beacon communication
as well, it ensures that the student is still in the classroom for the second
form of communication. Additionally, a timeout mechanism was implemented to
ensure that a student could not be outside of the classroom and simply send
another student in the classroom their code to send with their UUID in a forged
BLE beacon as this would take a substantial amount of coordination and time
between the two students that the timeout mechanism would likely catch. Along
with the timeout mechanism, a forgiveness mechanism was determined to be
necessary in case a message was lost or a student missed the timeout window
that generates a new code and sends that code via SMS communication three times
before determining that the student is not present.

This SMS second factor authentication further ensures the characteristic of
authentication accuracy and further supports the function of identity recording
by creating a 1-1 mapping of a student’s mobile phone to their identity.  This
second factor authentication also supports the characteristic of reliability
through the forgiveness mechanism. The forgiveness mechanism does not ensure
reliability but it does make consideration for the possibility of missed or
dropped messages. In tandem with the forgiveness mechanism is the timeout
mechanism. This mechanism provides support for the final function of attendance
tracking solutions of noting of time. This timeout mechanism ensures that the
time a student checks in is during the time of a course.

The dual authentication scheme for the BLEATS system leverages both BLE beacon
authentication through UUID as well as a one time code authentication through
SMS 1-1 mapping of a student to their mobile device. This authentication scheme
supports the desired attendance tracking characteristics of authentication
accuracy and reliability  while supporting the attendance tracking functions of
identity recording and time notation.

\subsection{Simulation} 

The two measures of BLE beaconing that we are most interested in investigating
for BLEATS are reliability and scale. More specifically, we want to determine:

\begin{enumerate}

\item what scale of device density (i.e. number of attendees in a BLEATS
system) can be achieved while maintaining reliable transmission

\item what scale of device distance (i.e. size of the room that implements a
BLEATS system) can be achieved while maintaining reliable transmission 

\end{enumerate}

To quantify these characteristics, we decide to measure the percentage of
successfully received packets out of total packets transmitted as the metric
for reliability of the BLEATS system. There are several interrelated variables
that are of interest. First, the distance between beaconing node and receiver
affects signal propagation strength and interference patterns. We expect that
increasing the distance between beacon advertising devices and the central base
receiving station will affect successful reception by reducing the strength of
the logarithmically-attenuating received signal and increasing the opportunity
for interference. Second, the number of devices affects the number of packet
collisions, since BLE's TDMA beaconing scheme has finite capacity. We thus
expect that increasing the number of beacon advertising devices will increase
the likelihood of wireless interference and lost packets. Third, the topology
of devices relative to the single receiving station will affect interference
patterns and also dictate the distance of devices. 

Device distance, quantity, and topology are heavily intertwined. In order to
best evaluate permissible operating ranges for BLEATS we will first evaluate
the effect of distance on a single device, to get an upper bound for the
distance that we can expect the devices to work at. We will then evaluate
different topologies with different numbers of device to determine the
conditions under which BLEATS will be able to reasonably operate under. 

Beyond distance, number of devices and topology, we can control the beacon
interval length (anywhere from 20 ms to 10 seconds) and the total number of
intervals that the receiver listens for. These parameters will affect the
overall capacity and reliability of the system, but instead of introducing
additional degrees of complexity and variability, we fix these parameters to
sane defaults. We note that the product of interval length and number of
intervals scanned yields the total scan length. Because BLEATS is focused on
classroom attendance tracking, where we measure the presence of captive,
immobile users who remain in attendance on the order of 10-100 minutes, we cap
the total scan length at 1 minute. For studying other, more mobile, or more
real-time attendance tracking systems it would be useful to investigate the
effects of modulating beacon interval length and total number of intervals
listened to by the receiver.

Initial results indicate that for a beacon interval of 10 seconds with 6 total
intervals, BLEATS can support over 1000 devices with high reliability (i.e. all
nodes successfully transmit to the central receiver at least once during the
60-second receiver scan). However, when performing simulations, systems with
more than 500 devices are extremely resource intensive and slow. This makes it
difficult to measure absolute upper-bound numbers for the BLEATS system. Instead
of simulating a 60 second interval with thousands of devices, we instead focus
on measuring a microcosm of BLEATS, receiving 6 100 millisecond intervals. This
allows us to gather much more robust simulation data, while still accurately
observing device quantity trends and optimal topologies.

