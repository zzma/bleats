\section{Introduction}

\label{sec:introduction}

Classroom attendance tracking (CAT) is an important practice at many
educational institutions for the purpose of evaluating student participation
and effort. It is also used as an accountability and security measure for
monitoring young children. Despite its broad utility and widespread prominence, CAT
remains a surprisingly time-consuming, manual, and error prone process that
does not scale well. In typical cases, attendance is tracked by an
instructor on a student-by-student basis and recorded by hand onto a paper log
or computer interface. This model is very error prone, and does not scale well
to large groups and lecture-style classroom environments. To solve these issues, many
systems have been proposed in recent years to improve the speed, deployment
ease, authentication accuracy, reliability, and scale of CAT by making use of a
range of technologies including RFID~\cite{kassim2012web}, biometric
fingerprint and voice authentication~\cite{reda2011hyke,taxila2009development},
and simple web application automation~\cite{akhila2013novel}. From our
evaluation, however, we find that these applications typically improve only one
or two of the key metrics previously mentioned. 

We propose a new CAT system called Bluetooth Low Energy Attendance Tracking
System (BLEATS) that uses Bluetooth Low Energy (BLE) beaconing on mobile
devices to address all five desired characteristics of current CAT systems.  

BLEATS works through individual student mobile devices communicating a unique
id via BLE beaconing to a central attendance tracker device, which may be
either mobile or fixed within the room. The central attendance tracker
automatically registers attendance and then performs additional authentication
steps to further verify attendance. The system design is straightforward, and
manages to address  many of the existing challenges typically found in CAT
systems. Computer automation makes it trivial to register multiple attendances
at the same time, which improves the speed of attendance tracking and enables
larger numbers of individuals to be tracked. The growing ubiquity of
BLE-capable mobile devices makes deployment of BLEATS more and more convenient,
without the need for additional specialized hardware. Finally, authentication
accuracy is strengthened through statistically-robust data transmission and the
addition of second factor authentication schemes.  

In this paper, we implement an prototype version of BLEATS and focus our
efforts on the investigation of different authentication schemes to provide
stronger authentication accuracy guarantees than simply receiving the proper
student-specific id via broadcast, which is easily spoofed. At the same time,
we simulate a BLEATS classroom environment to measure beacon transmission range
and device density limits for certain topologies. The simulations only model
ideal operating conditions for BLEATS, and real deployment environments are
much more complex and interference rich, so we use simulation as a means of
determining upper-bounds for the properties of interest. 

More concretely, our work on BLEATS provides the following contributions:
\begin{itemize}

\item A BLEATS base implementation on a laptop and Raspberry Pi's to track
classroom attendance.  

\item A dual authentication scheme for use in BLEATS to combat simple spoofing
attempts and ensure student identity and proximity for attendance tracking.  

\item A proposed authentication scheme to further build confidence in student
attendance.  

\item An exploration of the feasibility and scale limitations of the BLEATS
system via NS3 network simulation that suggests scalability over a thousand
attendees within a large attendance venue. 

\end{itemize}
